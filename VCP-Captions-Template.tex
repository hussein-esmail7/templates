% [FILENAME]
% Author: [AUTHOR]
% Created: [DATE]
% Updated: [DATE]
% Description: [DESCRIPTION]

% Description of THIS file: This is a LaTeX (pronounced `Lah-tech') template document for Accessibility Performance Captions because typing the script in LaTeX is faster than centering text in PowerPoint in each slide. 

% What is LaTeX? definition from https://www.latex-project.org/about
% LaTeX, which is pronounced «Lah-tech» or «Lay-tech» (to rhyme with «blech» or «Bertolt Brecht»), is a document preparation system for high-quality typesetting. It is most often used for medium-to-large technical or scientific documents but it can be used for almost any form of publishing.

% Why use LaTeX instead of PowerPoint?
% 1. You don't have to center text manually, LaTeX does it by itself when you tell it to.
% 2. LaTeX doesn't require a Microsoft Office licence to use. It's free for everyone (and part of the Free Open Source Software movement)!
% 3. If you have a table of contents with numbered sections and you add another one in the middle, everything is renumbered automatically
% 4. (The reason why it got me into it) It can be typed in any text editor (though I recommend using VSCode with the "LaTeX Workshop" extension by James Yu).

% This document is formatted so that you can follow along with these comments (starting with a `%') so you know what's happening.

% Here is the variables section where you can easily change what parameters are used (aspect ratio of the slides, background colour, text colour, text size, title, etc.)

% =========================== Variables ===========================
\newcommand{\myAspectRatio}{169}        % Aspect ratio
\newcommand{\myColorBackground}{black}  % Background colour
\newcommand{\myColorForeground}{white}  % Text colour
\newcommand{\myFontsize}{30pt}          % Font size
\newcommand{\myParagraphSpacing}{0em}   % Spacing between paragraphs
\newcommand{\myTitle}{Show Title Here}  % Title of show, on the first frame
% =========================== Variables ===========================

% You don't need to worry about this ``Configurations'' section. This is where most of the variable values are actually implemented (setting the colours, text size, etc.)

% ======================== Configurations ========================
\documentclass[professionalfont, aspectratio=\myAspectRatio]{beamer}
\title{\myTitle}
\usepackage[fontsize=\myFontsize]{fontsize} % Change font size
\usepackage{siunitx}
\usepackage{newtxtext,newtxmath}
\usepackage{bm}
\usepackage{parskip} % No indentation on new paragraphs
\usepackage[none]{hyphenat} % Used to wrap words when a word finishes instead of in the middle of a word
\setlength{\parskip}{\myParagraphSpacing} % Set spacing between paragraphs
\setbeamertemplate{navigation symbols}{} % No navigation symbols
\setbeamercolor{title}{fg=\myColorForeground, bg=\myColorBackground} % Only used for \maketitle
\setbeamercolor{normal text}{fg=\myColorForeground, bg=\myColorBackground}
\setbeamercolor{background}{bg=\myColorBackground}
% ======================== Configurations ========================

\begin{document} 
% Official beginning of the document. 
% Everything before \begin{document} is called the ``Preamble''

% Comments inside this document `environment' do not show in the compiled PDF

% ========================= Title slide ==========================
\begin{frame}       % Whatever is inside this is in one slide.
    \begin{center}  % Centers whatever text is inside
        \myTitle    % Prints variable for the show title
    \end{center}
\end{frame}
% ========================= Title slide ==========================

% CONTINUE HERE

% A normal slide looks like this (without the comment symbols)
% \begin{frame}
%     % Text here
% \end{frame}

% You may notice the text is indented here. This does not show up in the actual document. the issue with this is that the text is not centered. For that, we want to do something like this:
% \begin{frame}
%     \begin{center}
%         % Text here
%     \end{center}
% \end{frame}

% If you want to have a designated line break on a slide, I recommend using 2 backslashes ("\\"). In LateX, this means make a new line. here is an example of that:
% \begin{frame}
%     \begin{center}
%         This is some text. \\ This is on a new line.
%     \end{center}
% \end{frame}

% There are other special characters that you should be aware of, but it might be easier if you search them online as you go. Here are just some until you need to search yourself:
% 3 consecutive period dots (``...''): $\ldots$
% Quotation marks: `` for the beginning double quotation marks, and '' for the ending double quotation marks. Same for ` and '.
% Italic text: \textit{Text here}
% Bold text: \textbf{Text here}
% Underlined: \underline{Text here}

% New paragraph: 2 empty lines in between. Ex: If the last word is on line 30, line 31 must be empty, and the new paragraph starts on line 32. If there is text on line 31, it looks as if it's in the same paragraph (I would also use this to your advantage here for readability while typing).

% For italic, bold, underline, etc., if you want multiple, these commands can be placed inside one another (order doesn't matter). Example: 
% \textbf{\underline{This sentence is bolded and underlined.}}

\begin{frame}
    \begin{center}
        Start typing the script lines here!
    \end{center}
\end{frame}

\begin{frame}
    \begin{center}
        
    \end{center}
\end{frame}

\end{document} % Official end of the document. 
