% [FILENAME]
% Author: [AUTHOR]
% Created: [DATE]
% Updated: [DATE]
% Description: [DESCRIPTION]

% This part is used for https://github.com/hussein-esmail7/template-maker
% templateDescription: LaTeX Course Lecture Notes Document

% This file can be found at https://github.com/hussein-esmail7/templates

% =========================== Variables ===========================
\newcommand{\myAuthor}              {[AUTHOR]} % Author of the show
\newcommand{\mySubject}             {PDF Subject Here}  % Used for Metadata
\newcommand{\myKeywords}            {PDF Keywords Here} % Separated by comma
\newcommand{\myCourseDateCreated}   {Created Date Here}
\newcommand{\myCourseDateUpdated}   {\today}
\newcommand{\myCourseCredits}       {Credits num}
\newcommand{\myCourseCode}          {Course Code Here}
\newcommand{\myCourseTitle}         {Course Title here}
\newcommand{\myCourseProf}          {Course Prof Here}    % First Last only
\newcommand{\myCourseSemester}      {Year and season here} % Ex. 2019F 
\newcommand{\myCourseSchedule}      {Weekdays/Times of Day Here} % Ex. MWF 9:00
\newcommand{\myCourseSection}       {Section Here} % Single letter
\newcommand{\myCourseLocation}      {Course Location}
\newcommand{\myTitle}{\myCourseCode{} \myCourseCredits{}: \myCourseTitle{} \\ \myCourseSemester{} - \myCourseSchedule{}}
% =========================== Variables ===========================

% =========================== Shortcuts ===========================
\newcommand{\comment}[1]{}                          % Multiline comments
\newcommand{\image}[1]{[\textbf{IMAGE MISSING #1}]} % #1 is the descriptor about the image
\newenvironment{itemize*}{\begin{itemize}\setlength{\itemsep}{0pt}\setlength{\parskip}{0pt}}{\end{itemize}} % Bullet list
\newenvironment{enumerate*}{\begin{enumerate}\setlength{\itemsep}{0pt}\setlength{\parskip}{0pt}}{\end{enumerate}} % Numbered list
\newenvironment{enumalph*}{\begin{enumerate}[label=\alph*.]\setlength{\itemsep}{0pt}\setlength{\parskip}{0pt}}{\end{enumerate}} % Alphebetized list (lowercase)
% =========================== Shortcuts ===========================

\documentclass{article}
\title{\myTitle}
\author{\myAuthor}
\date{\myCourseDateUpdated}

\usepackage{hyperref}   % Used for adding PDF metadata
\usepackage{listings}   % Used for blocks of code
\usepackage{graphicx}   % Used for adding images
\usepackage{enumitem}	% Used for alphebetized lists

\graphicspath{{./}}     % Import images that are in the same folder as this .tex file

\hypersetup{
    colorlinks=true,
	urlcolor=blue,		% URL colors (\href{}{})
	linkcolor=black,	% TOC colors
    pdfborder={0 0 0},
    pdftitle={\myTitle},
    pdfauthor={\myAuthor},
    pdfsubject={\mySubject},
    pdfkeywords={\myKeywords}
}

\lstset{        % Coding block configuration
  basicstyle=\ttfamily,
  frame=single,
  breaklines=true
}

% Prevent word hyphenation
\tolerance=1
\emergencystretch=\maxdimen
\hyphenpenalty=10000
\hbadness=10000

\begin{document}        % Official beginning of the document.
\pagenumbering{roman}   % Use roman numbers before content
\maketitle              % Make title page
\newpage                % Insert a page break
\tableofcontents        % Makes TOC
\newpage                % Insert a page break
\pagenumbering{arabic}  % Brings the normal numbers back

\section{Introduction}
This PDF is the collection of my lecture notes for \myCourseCode{} that took
place during the \myCourseSemester{} semester. This was taught by
\myCourseProf{} for section \myCourseSection{}. This occurs on:
\myCourseSchedule{} at \myCourseLocation{}.

At the time of writing this document, all URLs in this document are correct.
Should you have any issues, please
\underline{\href{mailto:HusseinEsmailContact@gmail.com}{feel free to send me an
email}}. 

\section{How This Guide is Formatted}
This guide is made using \LaTeX{}, a typesetting document format, mainly used
for large documents like STEM documents, theses, and research papers. To read
more about what I think about \LaTeX{}, you
\href{https://husseinesmail.xyz/articles/is-latex-better.html}{can go here}.

% TODO: Lecture notes here

\comment{   % Lecture template
\section{Lecture NUM: DATE}
\begin{itemize*}
    \item NOTES HERE
\end{itemize*}
}

\comment{   % Tutorial template
\section{Tutorial NUM: DATE}
\begin{itemize*}
    \item NOTES HERE
\end{itemize*}
}

\comment{   % Lab template
\section{Lab NUM: DATE}
\begin{itemize*}
    \item NOTES HERE
\end{itemize*}
}

\end{document} % Official end of the document.
