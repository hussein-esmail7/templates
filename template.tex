% [FILENAME]
% Author: [AUTHOR]
% Created: [DATE]
% Updated: [DATE]
% Description: [DESCRIPTION]

% This part is used for https://github.com/hussein-esmail7/template-maker
% templateDescription: 1 - Default LaTeX Template

% This file can be found at https://github.com/hussein-esmail7/templates

% =========================== Variables ===========================
\def\myAuthor			{[AUTHOR]} % Author of this file
\def\mySubject			{PDF Subject Here}  % Used for Metadata
\def\myKeywords			{PDF Keywords Here} % Separated by comma
\def\myDateCreated		{[DATE]}
\def\myDateUpdated		{\today}
\def\myTitle			{[FILE FRONT]}
% =========================== Variables ===========================

% =========================== Shortcuts ===========================
\def\bg{backgroundcolor}
\def\len{\setlength}
\newcommand{\image}[1]{[\textbf{IMAGE MISSING #1}]} % #1 is the descriptor about the image
\newenvironment{itemize*}{\begin{itemize}\setlength{\itemsep}{0pt}\setlength{\parskip}{0pt}}{\end{itemize}}
\newenvironment{enumerate*}{\begin{enumerate}\setlength{\itemsep}{0pt}\setlength{\parskip}{0pt}}{\end{enumerate}}
\newenvironment{enumalph*}{\begin{enumerate}[label=\alph*.]\setlength{\itemsep}{0pt}\setlength{\parskip}{0pt}}{\end{enumerate}}
\newenvironment{enumq*}{\begin{enumerate}[label=Q{{\arabic*}}.]\setlength{\itemsep}{0pt}\setlength{\parskip}{0pt}}{\end{enumerate}}
\newcommand{\TODO}[1]{\todo[backgroundcolor=none, linecolor=red]{#1}}
\newcommand{\TODOimg}[1]{\todo[inline, \bg=gray]{\textbf{IMG}: #1}}
\newcommand{\TODOcontent}[1]{\todo[\bg=red, linecolor=red]{\textbf{CON}: #1}}
\newcommand{\TODOfig}[1]{\todo[inline, \bg=orange]{\textbf{FIG}: #1}}
% =========================== Shortcuts ===========================

\documentclass{article}
\title{\vspace{-3cm} \\ \myTitle}
\author{\myAuthor}
\date{\myDateUpdated}

\usepackage[colorinlistoftodos]{todonotes}	% Used for "to do" notes
\usepackage{amssymb}	% Used for math symbols
\usepackage{comment}	% Used for the 'comment' environment (multi-line)
\usepackage{enumitem}	% Used for enumalpha environment
\usepackage{float}		% Used for ``H'' flag in figure environment
\usepackage{graphicx}   % Used for adding images
\usepackage{listings}   % Used for blocks of code
\usepackage{longtable}	% Used to format longtables properly
\usepackage{multirow}
\usepackage{qrcode}
\usepackage{xcolor}

\usepackage{hyperref}   % Used for adding PDF metadata

\graphicspath{{./}}     % Import images in the same folder as this .tex file

\hypersetup{
    colorlinks=true,
	urlcolor=black,		% URL colors (\href{}{}). Changed to blue later
	linkcolor=black,	% Table of Contents text color. Used for TOC body only
    pdfborder={0 0 0},
    pdftitle={\myTitle},
    pdfauthor={\myAuthor},
    pdfsubject={\mySubject},
    pdfkeywords={\myKeywords}
}

\lstset{        % Coding block configuration
  basicstyle=\ttfamily,
  frame=single,
  breaklines=true
}

% Prevent word hyphenation
\tolerance=1
\emergencystretch=\maxdimen
\hyphenpenalty=10000
\hbadness=10000

\begin{document}        % Official beginning of the document.
\pagenumbering{roman}   % Use roman numbers before content
\maketitle              % Make title page
\begin{center}
	\qrcode[hyperlink]{}
\end{center}
\tableofcontents        % Makes TOC
% \newpage
% \listoffigures			% Makes ``List of Figures''
\todototoc				% Add "Todo list" to main TOC
\listoftodos			% Show all todo notes (from todonotes package) as index
\hypersetup{linkcolor=blue} % Any links to other sections are now blue
\newpage                % Insert a page break
\pagenumbering{arabic}  % Brings the normal numbers back
\flushleft

\section{Introduction}
\TODO{Purpose of document}

At the time of writing this document, all URLs in this document are correct.
Should you have any issues, please
\underline{\href{mailto:HusseinEsmailContact@gmail.com}
{feel free to send me an email}}.

\section{How This Guide is Formatted}
This guide is made using \LaTeX{}, a typesetting document format, mainly used
for large documents like STEM documents, theses, and research papers. To read
more about what I think about \LaTeX{}, you
\href{https://husseinesmail.xyz/articles/is-latex-better.html}{can go here}.

\begin{comment}
	TODO commands (using the todonotes package):
	\TODO{}			-> Normal
	\TODOimg{}		-> Insert image
	\TODOcontent{}	-> Insert more content here
	\TODOfig{}		-> Insert figure
\end{comment}
\end{document}
