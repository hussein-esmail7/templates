% [FILENAME]
% Author: [AUTHOR]
% Created: [DATE]
% Updated: [DATE]
% Description: [DESCRIPTION]

% This part is used for https://github.com/hussein-esmail7/template-maker
% templateDescription: Default LaTeX Template

% This file can be found at https://github.com/hussein-esmail7/templates

% =========================== Variables ===========================
\newcommand{\myAuthor}              {[AUTHOR]} % Author of this file
\newcommand{\mySubject}             {PDF Subject Here}  % Used for Metadata
\newcommand{\myKeywords}            {PDF Keywords Here} % Separated by comma
\newcommand{\myCourseDateCreated}   {[DATE]}
\newcommand{\myCourseDateUpdated}   {\today}
\newcommand{\myTitle}               {[FILE FRONT]}
% =========================== Variables ===========================

% =========================== Shortcuts ===========================
\newcommand{\comment}[1]{}                          % Multiline comments
\newcommand{\image}[1]{[\textbf{IMAGE MISSING #1}]} % #1 is the descriptor about the image
\newenvironment{itemize*}{\begin{itemize}\setlength{\itemsep}{0pt}\setlength{\parskip}{0pt}}{\end{itemize}}
\newenvironment{enumerate*}{\begin{enumerate}\setlength{\itemsep}{0pt}\setlength{\parskip}{0pt}}{\end{enumerate}}
% =========================== Shortcuts ===========================

\documentclass{article}
\title{\myTitle}
\author{\myAuthor}
\date{\myCourseDateUpdated} 

\usepackage{hyperref}   % Used for adding PDF metadata
\usepackage{listings}   % Used for blocks of code
\usepackage{graphicx}   % Used for adding images
\usepackage{listings}   % Used for blocks of code

\graphicspath{{./}}     % Import images that are in the same folder as this .tex file

\hypersetup{
    colorlinks=false, 
    pdfborder={0 0 0},
    pdftitle={\myTitle}, 
    pdfauthor={\myAuthor}, 
    pdfsubject={\mySubject}, 
    pdfkeywords={\myKeywords}
}

\lstset{        % Coding block configuration
  basicstyle=\ttfamily,
  frame=single,
  breaklines=true
}

\begin{document}        % Official beginning of the document. 
% \pagenumbering{roman}   % Use roman numbers before the start of the script so that page numbers are the same as the real script
\maketitle              % Make title page
\newpage                % Insert a page break
\tableofcontents        % Makes TOC
\newpage                % Insert a page break
% \pagenumbering{arabic}  % Brings the normal numbers back

\section{Introduction}
% TODO: Purpose of this document here

At the time of writing this document, all URLs in this document are correct. 
Should you have any issues, please 
\underline{\href{mailto:HusseinEsmailContact@gmail.com}
{feel free to send me an email}}.

\section{How This Guide is Formatted}
This guide is made using \LaTeX{}, a typesetting document format, mainly used 
for large documents like STEM documents, theses, and research papers. To read 
more about what I think about \LaTeX{}, you 
\href{https://husseinesmail.xyz/articles/is-latex-better.html}{can go here}.

\end{document}
