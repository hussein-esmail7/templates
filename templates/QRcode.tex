% [FILENAME]
% Author: [AUTHOR]
% Created: [DATE]
% Updated: [DATE]
% Description: [DESCRIPTION]

% This part is used for https://github.com/hussein-esmail7/template-maker
% templateDescription: 6 - QR Code Poster in LaTeX

% This file can be found at https://github.com/hussein-esmail7/templates

% =========================== Variables ===========================
\def\myTitle			{[FILE FRONT]}
\documentclass{article}
\title{\vspace{-3cm} \\ \myTitle}
\date{ } % Must set to empty or else it will display the current date

% If you want the date to show up on the document
% \date{[DATE]}

\usepackage{qrcode}		% Used for generating the QR code
\usepackage{hyperref}   % Used for adding PDF metadata

\hypersetup{
    colorlinks=true,
	urlcolor=black,		% URL colors (\href{}{}) and QR codes of URLs
	linkcolor=black,	% Table of Contents text color. Used for TOC body only
    pdfborder={0 0 0},
    pdftitle={\myTitle},
    pdfauthor={\myAuthor},
    pdfsubject={\mySubject},
    pdfkeywords={\myKeywords}
}

\begin{document}        % Official beginning of the document.
\maketitle              % Make title page
\pagenumbering{gobble}  % Sets to no page numbering
% \pagenumbering{arabic}  % Brings the normal numbers back

\hypersetup{linkcolor=black} % Any links to other sections are now black
\centering

% This version for a black QR code and no hyperlink
\qrcode[nolink, height=3in]{Test} % QR Code contents in the {}

% Normal version
% \qrcode[height=3in]{Test} % QR Code contents in the {}

\end{document}        % Official beginning of the document.
