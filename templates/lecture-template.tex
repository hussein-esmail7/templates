% [FILENAME]
% Author: [AUTHOR]
% Created: [DATE]
% Updated: [DATE]
% Description: [DESC]

% This part is used for https://github.com/hussein-esmail7/template-maker
% templateDescription: 2 - LaTeX School Course Lecture Notes Document

% This file can be found at https://github.com/hussein-esmail7/templates

% =========================== Variables ===========================
\def\myAuthor              {[AUTHOR]} % Author of the show
\def\mySubject             {[SUBJ]}
\def\myKeywords            {[KEYWORDS]} % Separated by comma
\def\myCourseDateCreated   {[DATE]}
\def\myCourseDateUpdated   {\today}
\def\myCourseCredits       {[COURSE-CREDITS]}
\def\myCourseCode          {[COURSE-CODE]}
\def\myCourseTitle         {[COURSE-TITLE]}
\def\myCourseProf          {[COURSE-PROF]}    % First Last only
\def\Prof				   {[PROF]}    % First name only
\def\myCourseSemester      {[COURSE-SEMESTER]} % Ex. 2019F
\def\myCourseSchedule      {[COURSE-SCHEDULE]} % Ex. MWF 9:00
\def\myCourseSection       {[COURSE-SECTION]} % Single letter
\def\myCourseLocation      {[COURSE-LOCATION]}
\def\myTitle{\myCourseCode{} \myCourseCredits{}: \myCourseTitle{} \\ \myCourseSemester{} - \myCourseSchedule{}}
% =========================== Variables ===========================

% =========================== Shortcuts ===========================
\def\bg{backgroundcolor}
\def\len{\setlength}
\newcommand{\image}[1]{[\textbf{IMAGE MISSING #1}]} % #1 is the descriptor about the image
\newenvironment{itemize*}{\begin{itemize}\setlength{\itemsep}{0pt}\setlength{\parskip}{0pt}}{\end{itemize}}
\newenvironment{enumerate*}{\begin{enumerate}\setlength{\itemsep}{0pt}\setlength{\parskip}{0pt}}{\end{enumerate}}
\newenvironment{enumalph*}{\begin{enumerate}[label=\alph*.]\setlength{\itemsep}{0pt}\setlength{\parskip}{0pt}}{\end{enumerate}}
\newenvironment{enumq*}{\begin{enumerate}[label=Q{{\arabic*}}.]\setlength{\itemsep}{0pt}\setlength{\parskip}{0pt}}{\end{enumerate}}
\newcommand{\TODO}[1]{\todo[backgroundcolor=none, linecolor=red]{#1}}
\newcommand{\TODOimg}[1]{\todo[inline, \bg=gray]{\textbf{IMG}: #1}}
\newcommand{\TODOcontent}[1]{\todo[\bg=red, linecolor=red]{\textbf{CON}: #1}}
\newcommand{\TODOfig}[1]{\todo[inline, \bg=orange]{\textbf{FIG}: #1}}
% =========================== Shortcuts ===========================

\documentclass{article}
\title{\vspace{-3cm} \\ \myTitle}
\author{\myAuthor}
\date{\myCourseDateUpdated}

\usepackage[colorinlistoftodos]{todonotes}	% Used for "to do" notes
\usepackage{amssymb}	% Used for math symbols
\usepackage{comment}	% Used for the 'comment' environment (multi-line)
\usepackage{enumitem}	% Used for enumalpha environment
\usepackage{float}		% Used for ``H'' flag in figure environment
\usepackage{graphicx}   % Used for adding images
\usepackage{listings}   % Used for blocks of code
\usepackage{longtable}	% Used to format longtables properly
\usepackage{multirow}
\usepackage{xcolor}

\usepackage{hyperref}   % Used for adding PDF metadata

\graphicspath{{./}}     % Import images that are in the same folder as this .tex file

\hypersetup{
    colorlinks=true,
	urlcolor=blue,		% URL colors (\href{}{})
	linkcolor=black,	% Table of Contents text color. Used for TOC body only
    pdfborder={0 0 0},
    pdftitle={\myTitle},
    pdfauthor={\myAuthor},
    pdfsubject={\mySubject},
    pdfkeywords={\myKeywords}
}

\lstset{        % Coding block configuration
  basicstyle=\ttfamily,
  frame=single,
  breaklines=true
}

% Prevent word hyphenation
\tolerance=1
\emergencystretch=\maxdimen
\hyphenpenalty=10000
\hbadness=10000

\begin{document}        % Official beginning of the document.
\pagenumbering{roman}   % Use roman numbers before content
\maketitle              % Make title page
\tableofcontents        % Makes TOC
\newpage
\listoffigures			% Makes ``List of Figures''
\todototoc				% Add "Todo list" to main TOC
\listoftodos			% Show all todo notes (from todonotes package) as index
\hypersetup{linkcolor=blue} % Any links to other sections are now blue
\newpage                % Insert a page break
\pagenumbering{arabic}  % Brings the normal numbers back

\section{Introduction}
This PDF is the collection of my lecture notes for \myCourseCode{} that took
place during the \myCourseSemester{} semester. This was taught by
\myCourseProf{} for section \myCourseSection{}. This occurs on:
\myCourseSchedule{} at \myCourseLocation{}.

At the time of writing this document, all URLs in this document are correct.
Should you have any issues, please
\underline{\href{mailto:HusseinEsmailContact@gmail.com}{feel free to send me an
email}}.

\section{How This Guide is Formatted}
This guide is made using \LaTeX{}, a typesetting document format, mainly used
for large documents like STEM documents, theses, and research papers. To read
more about what I think about \LaTeX{}, you
\href{https://husseinesmail.xyz/articles/is-latex-better.html}{can go here}.

% TODO: Lecture notes here

\begin{comment}
% Lecture template
\section{Lecture NUM: DATE}
\begin{itemize*}
    \item NOTES HERE
\end{itemize*}
\end{comment}

\begin{comment}
% Tutorial template
\section{Tutorial NUM: DATE}
\begin{itemize*}
    \item NOTES HERE
\end{itemize*}
\end{comment}

\begin{comment}
% Lab template
\section{Lab NUM: DATE}
\begin{itemize*}
    \item NOTES HERE
\end{itemize*}
\end{comment}

\begin{comment}
	TODO commands (using the todonotes package):
	\TODO{}			-> Normal
	\TODOimg{}		-> Insert image
	\TODOcontent{}	-> Insert more content here
	\TODOfig{}		-> Insert figure
\end{comment}
\end{document} % Official end of the document.
